\documentclass[a4paper,12pt]{article}
\setlength{\parindent}{0in}
\usepackage{booktabs}
\usepackage{dsfont}
\usepackage{float}
\usepackage{amssymb}

\title{Problem Sheet \#3}
\author{Aayush Sharma Acharya}
\date{\today}

\begin{document}
    \maketitle
    \textbf{Problem 3.1: } \textit{Distributive laws for sets}\\

    Let A, B and C be sets. Proof that the following two distributive laws hold:

    $$ \textit{A} \cup (\textit{B} \cap \textit{C}) = ( \textit{A} \cup \textit{B} ) \cap ( \textit{A} \cup \textit{C} )$$
    \begin{center}
        and,
    \end{center}
    $$ \textit{A} \cap (\textit{B} \cup \textit{C}) = ( \textit{A} \cap \textit{B} ) \cup ( \textit{A} \cap \textit{C} )$$

    \textit{Solution},

    \indent Let \textit{P} and \textit{Q} be any two sets:

    Now,

    $$\textit{P} \cup \textit{Q} := \{ x: x \in \textit{P} \  or\  \textit{x} \in \textit{Q} \}$$

    And,
    $$\textit{P} \cap \textit{Q} := \{ x: x \in \textit{P} \ and\  \textit{x} \in \textit{Q} \}$$

    Let T be the event when an element \textit{x} is present in the set.
    \begin{center}
        and
    \end{center}

    F be the event when an element \textit{x} is NOT present in the set.

    Recalling the definition of $\textit{P} \cap \textit{Q}$ and $\textit{P} \cup \textit{Q}$, we can construct a table in order to prove the relation:
    $$\textit{A} \cup (\textit{B} \cap \textit{C}) = ( \textit{A} \cup \textit{B} ) \cap ( \textit{A} \cup \textit{C} )$$
    \begin{table}[H]
        \begin{center}
            \begin{tabular}{| l | l | l | l | l | l | l | l |}
                \midrule
                \textit{A}& \textit{B}& \textit{C} & $\textit{A} \cup \textit{B}$ & $\textit{A} \cup \textit{C}$ & \textbf{$(\textit{A} \cup \textit{B}) \cap (\textit{A} \cup \textit{C})$} & $\textit{B} \cap \textit{C}$ & \textbf{$\textit{A} \cup (\textit{B} \cap \textit{C})$}\\
                \midrule\midrule
                T & T & T & T & T & \textbf{T} & T & \textbf{T}\\
                \midrule
                T & T & F & T & T & \textbf{T} & F & \textbf{T}\\
                \midrule
                T & F & T & T & T & \textbf{T} & F & \textbf{T}\\
                \midrule
                T & F & F & T & T & \textbf{T} & F & \textbf{T}\\
                \midrule
                F & T & T & T & T & \textbf{T} & T & \textbf{T}\\
                \midrule
                F & T & F & T & F & \textbf{F} & F & \textbf{F}\\
                \midrule
                F & F & T & F & T & \textbf{F} & F & \textbf{F}\\
                \midrule
                F & F & F & F & F & \textbf{F} & F & \textbf{F}\\
                \midrule
            \end{tabular}
        \end{center}
    \end{table}
    Here in this table we have generated a full list of possible events when \textit{x} is present and not present in both set \textit{A} and \textit{B} and we have found that the events in $( \textit{A} \cup \textit{B} ) \cap ( \textit{A} \cup \textit{C} )$ and $\textit{A} \cup (\textit{B} \cap \textit{C})$ are equivalent.
    This says that the distributive laws hold for the following:
    $$ \textit{A} \cup (\textit{B} \cap \textit{C}) = ( \textit{A} \cup \textit{B} ) \cap ( \textit{A} \cup \textit{C} )$$
    Using the same method we can construct another table for the following:
    $$ \textit{A} \cap (\textit{B} \cup \textit{C}) = (\textit{A} \cap \textit{B}) \cup (\textit{A} \cap \textit{C}) $$

    \begin{table}[H]
        \begin{center}
            \begin{tabular}{| l | l | l || l | l | l | l | l |}
                \midrule
                \textit{A}& \textit{B}& \textit{C} & $\textit{A} \cap \textit{B}$ & $\textit{A} \cap \textit{C}$ & \textbf{$(\textit{A} \cap \textit{B}) \cup (\textit{A} \cap \textit{C})$} & $\textit{B} \cup \textit{C}$ & \textbf{$\textit{A} \cap (\textit{B} \cup \textit{C})$}\\
                \midrule\midrule
                T & T & T & T & T & \textbf{T} & T & \textbf{T}\\
                \midrule
                T & T & F & T & F & \textbf{T} & T & \textbf{T}\\
                \midrule
                T & F & T & F & T & \textbf{T} & T & \textbf{T}\\
                \midrule
                T & F & F & F & F & \textbf{F} & F & \textbf{F}\\
                \midrule
                F & T & T & F & F & \textbf{F} & T & \textbf{F}\\
                \midrule
                F & T & F & F & F & \textbf{F} & T & \textbf{F}\\
                \midrule
                F & F & T & F & F & \textbf{F} & T & \textbf{F}\\
                \midrule
                F & F & F & F & F & \textbf{F} & F & \textbf{F}\\
                \midrule
            \end{tabular}
        \end{center}
    \end{table}
    In this table we have generated a full list of possible events when \textit{x} is present and not present in both set \textit{A} and \textit{B} and we have found that the events in $( \textit{A} \cap \textit{B} ) \cup ( \textit{A} \cap \textit{C} )$ and $\textit{A} \cap (\textit{B} \cup \textit{C})$ are equivalent.
    This, also, says that the distributive laws hold for the following:
    $$ \textit{A} \cap (\textit{B} \cup \textit{C}) = ( \textit{A} \cap \textit{B} ) \cup ( \textit{A} \cap \textit{C} )$$\\


    \textbf{Problem 3.2: } \textit{reflexive, symmetric and transitive}

    For each of the following relations, determine whether they are reflexive, symmetric, or transitive.Provide a reasoning.\\

    a) $\mathds{R} = \{(\textit{a},\textit{b}) | \textit{a}, \textit{b} \in \mathds{Z} \wedge \textit{a} \neq  \textit{b} \}$\\

    $\textit{a} \neq \textit{a}$ is not true hence it is not reflexive.

    If $\textit{a} \neq \textit{b}$ is true then $\textit{b} \neq \textit{a} $ is also true. Hence it is symmetric.

    If $ \textit{a} \neq \textit{b}$ is true and assume $ \textit{b} \neq \textit{c}$ is true, then $ \textit{a} \neq \textit{c}$ may or may not be true. So it is not transitive.\\

    b) $\mathds{R} = \{(\textit{a},\textit{b}) | \textit{a}, \textit{b} \in \mathds{Z} \wedge |\textit{a} - \textit{b} | \leq 3  \}$\\

    $ |\textit{a} - \textit{a} | \leq 3$ and $ |\textit{b} - \textit{b} | \leq 3$ is true, hence it is reflexive.

    If  $ |\textit{a} - \textit{b} | \leq 3$ is true, then  $ |\textit{b} - \textit{a} | \leq 3$ is also true. So it is also symmetric.

    If $ |\textit{a} - \textit{b} | \leq 3$ and lets assume that $ |\textit{b} - \textit{c} | \leq 3$, then $ |\textit{a} - \textit{c} | \leq 3$ may be greater than or less than 3 or sometimes equals to 3. Hence it is not transitive.\\

    c) $\mathds{R} = \{(\textit{a},\textit{b}) | \textit{a}, \textit{b} \in \mathds{Z} \wedge (\textit{a}\ mod\ 10) = (\textit{b}\ mod\ 10) \}$\\

    $(\textit{a}\ mod\ 10) = (\textit{a}\ mod\ 10)$ is always true and $(\textit{b}\ mod\ 10) = (\textit{b}\ mod\ 10)$ is also always true. Therefore, it is reflexive.

    If $(\textit{a}\ mod\ 10) = (\textit{b}\ mod\ 10)$ is True, then $(\textit{b}\ mod\ 10) = (\textit{a}\ mod\ 10)$ is also True. Hence, it is also symmetric.

    It also transitive because if $(\textit{a}\ mod\ 10) = (\textit{b}\ mod\ 10)$ and $(\textit{b}\ mod\ 10) = (\textit{c}\ mod\ 10)$ then, the relation $(\textit{a}\ mod\ 10) = (\textit{c}\ mod\ 10)$ is also true.\\


    \textbf{Problem 3.3:} \textit{Circular prime numbers (haskell)}

    \begin{center}
        Source code is also included in the zip file.
    \end{center}

    a) Implement a function \textbf{prime :: Integer -\textgreater\ Bool} that returns a Bool value indicating whether the Integer argument is a prime number or not.

    \begin{verbatim}
        prime :: Integer -> Bool
        prime n = length [x | x <- [1..n], n `mod` x == 0] == 2
    \end{verbatim}

    This function takes a number \textit{n} and finds its remainder by dividing it from \textit{1 to n}. For example \textit{n} = 92, It would look like:


    \begin{verbatim}
        91 `mod` 1  = 0
        91 `mod` 2  = 1
        91 `mod` 3  = 1
        91 `mod` 4  = 3
        .   .    .    .
        .   .    .    .
        .   .    .    .
        91 `mod` 91 = 0
    \end{verbatim}

    This would generate all the possible factors of \textit{n}. Then it would filter the list of numbers which has only 2 factors (which is why I used the length function). The number which has only 2 factors (1 and number itself) is called the prime number.

    We can test the validity of our function by:
    \begin{verbatim}
     > prime 2
     True
    > filter prime [2..100]
[2,3,5,7,11,13,17,19,23,29,31,37,41,43,47,53,59,61,67,71,73,79,83,89,97]
    \end{verbatim}

    b) Using the prime function, implement a function \textbf{circprime :: Integer -\textgreater Bool} that returns
    a \textbf{Bool} value indicating whether the Integer value is a circular prime number or not.

    \begin{verbatim}
        circprime :: Integer -> Bool
        circprime n =
                length [z | z <- [prime y |
                            y <- [read x :: Integer |
                            x <- circle (show n)]],
                    z == True]
                    == length (circle (show n))
    \end{verbatim}

    First of all the function \textbf{circprime} accepts a number \textbf{\textit{n}} and then converts it to string using \textbf{show} function. After it is converted it is then passed to \textbf{circle} function which I had written from the previous assignment.The circle function provides me with all the cyclic orders of the string. After all the orders of the string are returned in an array, I convert each combination string in the array to numbers and pass it to function \textbf{prime} which I had written earlier. It returns a list of True and False which denotes that combination of number is a prime number or not. If the list contains True values that is equal to the length of \textbf{(circle (show \textit{n}))}, that prime number is circular.
\end{document}