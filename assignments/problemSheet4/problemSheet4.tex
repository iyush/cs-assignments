\documentclass[a4paper,12pt]{article}
\setlength{\parindent}{0in}
\usepackage{booktabs}
\usepackage{dsfont}
\usepackage{float}
\usepackage{amssymb}

\title{Problem Sheet \#4}
\author{Aayush Sharma Acharya}
\date{\today}

\begin{document}
    \maketitle
    \textbf{Problem 4.1: } \textit{Prefix order relations}
    \\

    Let $\Sigma$ be a finite set (called an alphabet) and let $\Sigma *$ be the be the set of all words that can be created out the symbols in the alphabet $\Sigma$. ($\Sigma *$ is the Kleene closure of  $\Sigma$, which includes the empty word $\epsilon$). A word  $\textit{p}\in\Sigma *$ is called a prefix of a word $\textit{w}\in\Sigma *$ if there is a word $\textit{q}\in\Sigma *$ such that $\textit{w}=\textit{p}\textit{q}$. A prefix \textit{p} is called a proper prefix if $\textit{p} \neq \textit{w}$.
    \\

     a) Let $\preceq \subseteq \Sigma*$ x $ \Sigma *$ be a relation such that $\textit{p} \preceq \textit{w}$ for \textit{p}, \textit{w} $\in \Sigma$ if p is a prefix of w. Show that $\preceq$ is a partial order.\\

     Let's assume that the relation is partial order relation. We will see if it holds or not.\\

     In order for a relation to be a partial order. It must possess there properties:
     \newline a) The relation must be \textbf{reflexive}.
     \newline b) The relation must be \textbf{antisymmetric}.
     \newline c) The relation must be \textbf{transitive}.\\

    Relexive relation is defined by:
    \begin{center}
        A relation is reflexive if $\forall \textit{p} \in$ A, $x(p,p) \in$ R. where R is the universal set.
    \end{center}

    Here the relation is reflexive because it p




    Relexive relation is defined by:
    \begin{center}
        A relation is reflexive if $\forall \textit{a} \in$ A, $x(a,b) \in$ R. where R is the universal set.
    \end{center}

    Relexive relation is defined by:
    \begin{center}
        A relation is reflexive if $\forall \textit{a} \in$ A, $x(a,b) \in$ R. where R is the universal set.
    \end{center}




\end{document}